\documentclass{article}

\usepackage[T1]{fontenc}
\usepackage{MnSymbol}

\begin{document}

	\begin{center}
		\huge
		Software Agents:\\
		Project 1
	\end{center}

	\begin{flushright}
		\normalsize
		Joshua Torrance\\
		267920
	\end{flushright}


\section{Introduction}

The blocks world domain is a well known problem domain which do to its initial
simplicity has been used frequently in the study of AI and planning. Simple
blocks world consists of a set of blocks and a robots. The aim is to get the robot
to stack the blocks in a given order as fast as possible.

\section{Extensions to Blocks World}
This project had several goals all of which extend the simple blocks world domain
in various ways.

\subsection{Multiple Agents}

A common addition is multiple robots since this allows interesting examination
of multi-agent interactions and cooperation. One way of implementing multiple
agents is simply to get the agents to take turns doing actions and interleave
those actions in the action sequence. This simulates simultaneous actions but
it is difficult to get the robots to cooperate. A better way is to get the robots
to actually take their actions simultaneously.


\subsection{Block Weights}

Complexity is added to the simple blocks world domain by giving all of the
blocks a certain weights and the robots strength such that a robot is unable to
lift a block if it is too heavy for it. This means that in order to lift some heavy
blocks the robots must cooperate.

\subsection{Height}

Another simple extension is to give each of the robots a height such that they
are unable to stack block above their height. For example a robot with a height
of 1 would be unable to stack any blocks on top of other blocks.

\subsection{Agent Cooperation}

As mentioned above in order to lift heavy blocks the agents must cooperate. The
creates extra complexity for the problem in order to get the robots to cooperate
effectively.

\subsection{Dynamic Goals}

Instead of the user specifying the exact sequence of blocks they would like
it is possible to specify a set of rules that make a desirable outcome such as
minimising the number of stacks or putting all the heavy blocks on the bottom.

\section{Programs}

The results of this project consist of two programs which implement an extended
blocks world in two different ways. The frst program implement the domain
with a original implementation of the situation calculus with successful use of
simultaneous actions. The second program is built on ConGolog and generates
plans with dynamic goals but without true simultaneous actions.
Unfortunately both programs suffer from poor performance when given any
scenario with a moderate number of entities.

\subsection{sitcalc}
Sitcalc has extended blocks world to deal with:
\begin{itemize}
	\itemˆ Multiple agents.
	\itemˆ Block weight and robot strength.
	\itemˆ Robot height.
	\itemˆ True simultaneous actions.
	\itemˆ Agent cooperation.
	\itemˆ Simply dynamic goals.
\end{itemize}

In order to implement simultaneous, cooperative actions in stead of the
standard do(A,S) style of situations sitcalc uses a list of actions; do([ A,B,...],S)
combined with a set of rules (the simultaneous\_actions predicates) which define
the combinations of actions that are not permitted. This, along with the stan-
dard situation calculus, allows the agents to simultaneous perform cooperative
actions.

The new situation calculus has also been implemented independent of the
domain.

\subsection{ConGolog}
The ConGolog program implements similar features to the sitcalc program,
albeit slightly more polished, but is using Ryan Kelly's implementation of Con-
Golog for the situation calculus.
This program implements:
\begin{itemize}
	\itemˆ Multiple agents with interleaved actions.
	\itemˆ Block weight and robot strength.
	\itemˆ Robot height.
	\itemˆ Complicated dynamic goals.
\end{itemize}

The programs goals are generated during run time in order to be able to
create a reasonable goal for any initial state. A number of arbitrary rules are
used to generate possible goals:
\begin{itemize}
	\itemˆ All blocks must be part of the solution.
	\itemˆ Minimise the number of stacks.
	\itemˆ Place the heavier blocks at the bottom of the stacks.
\end{itemize}

The last component of goal generation is minimising the time required to
access each type of block. Each block is has a type (such as productA or
productB) and the goal with the lowest average access time for each type of
block is chosen as the final goal. This could be useful in a warehouse simulation
where quick access to types of good is required.


\section{Further Work}

There is obviously a huge amount of scope for the blocks world domain but the
main things that could benefit these program involve the execution times for
complicated situations (which are currently woeful for both programs). Fortu-
nately there is a large amount of literature dealing with this issue.
There are many other interesting extensions that could be implemented for
these programs:
\begin{itemize}
	\itemˆ Heuristic action choice.
	\itemˆ An ongoing warehouse problem where new blocks are added to the system
and certain blocks must be removed from the system at times (these would
be exogenous actions).
	\itemˆ More complicated handling of robot-block interactions (such as some robots
being able to handle certain types of blocks or robot movement around
blocks).
	\itemˆ Knowledge about block attributes and locations (imagine a warehouse
with multiple rooms and robots have limited knowledge about rooms they
aren't in).
\end{itemize}

\end{document}


